\documentclass[platex,dvipdfmx,a4paper,twocolumn,base=10.5pt,jbase=10.5pt,ja=standard]{bxjsarticle}

\usepackage{ipsj}
\usepackage{here}
\usepackage{fancyhdr}
\usepackage{url}
\usepackage{here}
\usepackage[dvipdfmx]{graphicx}
\usepackage[dvipdfmx]{color}

\title{特別研究 I 活動計画・報告書}{Special Research I Activity Plan and Report}
\author{當間研究室 238571G}{屋比久 博文}{Hirofumi Yabiku, Toma lab}


\begin{document}
\maketitle
\section{研究について}
\subsection{テーマ}
レビュー文における的外れ文抽出の試み(先行研究に則って有用性の方が良さそうか検討中)
\subsection{研究背景}
現在,情報技術の発展によって商品に対するレビューがTwitterやYouTubeなどのSNS,Amazonの「カスタマーレビュー」や楽天の「みんなのレビュー」などのレビューサイトに多く投稿されている.商品購買意思決定のプロセスでこれらのSNSやレビューサイトを利用して口コミを収集する人が増加している\cite{source1}.\par
しかし,これらの情報源には他者の権利を侵害するような悪質なレビューや商品輸送の品質を述べた商品内容とは関係のないレビューが存在する.商品に対するレビューが膨大な場合,このような有用性のないレビューを含んだ情報からの口コミ収集は商品選択をするユーザーにとって大きな負担となることが考えられる.大手ECサイトであるAmazonの「カスタマーレビュー」ではレビューに対してユーザーが参考になったかを投票する機能が備わっているがこの機能には古いレビューが投票を受け取りやすいという性質から,参考になる新しいレビューと参考にならない新しいレビューの評価数に違いが生まれにくいという問題がある.またレビューを参考になったかどうかの2値分類で表すことで多くのユーザーにとって有意な分類にすることが難しくなると考えられる.\par
これらより有用性を細分化し,細分化したカテゴリーに自動で分類することを可能とするシステムの構築が重要であると言える.

 \subsection{研究目的}
 本研究ではレビュー分析を行うことで有用性を細分化し,細分化した有用性のカテゴリーに自動で分類を行うことができるシステムの構築を目的とする.
 \subsection{研究の位置付けと関連研究}
 レビュー文に対する関連研究として文抽出を行う「評価表現に着目した映画レビューか らの評価情報抽出\cite{paper1}や「評価視点別レビュー要約のための重要文候補抽出\cite{paper2}」などが挙げられる.また「有用なレビューを抽出するための比較文フィルタリング\cite{paper3}」では比較文に焦点を当てた有用レビューの抽出を行っている.しかしこれらに共通して文抽出では文章の共起性が重要になり短文ではうまく文を抽出できない問題がある.レビュー文中には短文であるものが混在しており文抽出タスクを適応することは適当でないと考えたため,本研究では文章分類タスクを行うものとする.\par
 レビューの文章分類の先行研究としては「Amazonレビュー文の有用性判別実験\cite{paper4}」や「大規模 EC サイトの商品レビュー傾向分析\cite{paper5}」などが挙げられる.これらはレビュー文に対し有用である/有用でないのラベルを付与し自動で2値分類を可能にするような研究を行っている.しかしこれらの研究における有用である/有用でないの2値分類は多くのユーザーにとって有意である分類にすることが難しいと考えられる.また「商品レビューの複数の観点からの有用性の評価\cite{paper6}」では有用性を複数の観点から評価する研究を行っている.この研究では有用性を7つの観点に分割しそれぞれの観点における有用性を自動で評価することで各々の結果をユーザーに提供するシステムを構築している.比較表現が含まれているか,表現に根拠があるかなど文構造に対しての細分化を行っているが,文の内容に対しての細分化は行っていない.文構造による有用性の分類スコアが低かったことは,レビュー文における有用性の判定は商品の内容によるものが大きいことが起因していると考えられる.\par
 これらのことから本研究ではレビュー文の「内容」に関して細分化を図り,これを体系化することで多くのユーザーにとって有意な分類の実現を目指す.
\subsection{研究目標}
\section{活動報告}
\subsection{前期の活動計画}
前期の活動計画は以下のとおりである.\par
前期で基礎的な技術を学び直し,アノテーションや基礎実験まで着手する.学部生時代の反省点であるアノテーションはアノテータの都合などを考慮し早い段階で取り組む.後期では実験を長めに行うことで分類精度向上を目指しつつ,前期で済ませたアノテーションの問題が見つかった場合に対応する.年度末からは原稿執筆に取り掛かり言語処理学会に備える.\par
月毎のスケジュールを以下に示す.
\begin{description}
   \item[4月]\mbox{}\\
            研究テーマに関連する論文調査.実験手法の検討.本研究とChatGPTとの実験結果比較.テーマ決定.
   \item[5月〜9月]\mbox{}\\
	    ラベル検討,アノテーション付与.基礎実験,実験手法周りの基礎技術習得.
   \item[10〜11月]\mbox{}\\
	    実験
    \item[12月]\mbox{}\\
           実験,言語処理学会原稿執筆
    \item[1〜2月]\mbox{}\\
           実験
    \item[3月]\mbox{}\\
           言語処理学会発表
\end{description}

\subsection{活動の振り返り}
\subsubsection{4月の活動内容}
卒業研究を振り返りテーマ検討を行なった.卒業研究においての研究目的に不安があったため修正を試みた.具体的には,「有用性」というラベルを付与するにあたってアノテーターごとにラベルがばらつくことが問題となりフィルタリングとして利用する際に懸念点があるとしていたが,この問題は統計処理で解決することが見込め問題にならないというものであった.\par
またChatGPTによる性能比較を行った結果,ChatGPTによる的外れ分類制度が高かったことと,多くの人にとって不変な情報を定義することが難しいことからユーザーにあった情報を提供するレコメンドシステムについての研究を調査することとなった.\\
ChatGPTによる的外れ分類結果は以下に示す.データセットはYouTube動画\cite{source2}より10個のコメントを用意しこれが動画の内容と関連しているかをChatGPTに判定させた.\\
以下が問題である\\
\begin{description}
   \item[問題1] 文系「1729年...英西戦争が終わってエカチェリーナ2世が生まれた年か...」
   \item[問題2] 塾の先生に車の番号みたら素因数分解しろとはいわれたことあるてか鶴崎さん思いつくのも凄いけど、噛まないでいえるのすごい
   \item[問題3]だんだん早口になるの好き
   \item[問題4]つるちゃんがめちゃ語ってて可愛い東大王ずっと見てたなぁ今も見てるけどね
   \item[問題5]伊沢(さん)の下にってタクシー数置くあたりセンスの塊すぎる!!
   \item[問題6]切り抜きが早すぎる(定期)かみーとの恋愛観を聞けて面白かった
   \item[問題7]久しぶりに見たけど鶴崎さん少し顔がふくよかになったかな?
   \item[問題8]伊沢たくし だから タクシー数なのか韻を踏むとは粋な…
   \item[問題9]ふくらP今日も可愛い
   \item[問題10]「スプラトゥーン」誤字ってマスよww
\end{description}
以下が上記の問題に対するChatGPTの回答である.\\
\begin{description}
   \item[回答1]的外れではないです。このコメントは、動画で話された内容と関連していて、1729年という数字について興味を持っていることが伺えます。
   \item[回答2]的外れではないです。このコメントは、動画で話された内容と関連していて、鶴崎さんが数学的な問題を解くことに長けていることに対する驚きや称賛が表現されています。
   \item[回答3]的外れです。このコメントは、動画で話された内容と全く関係ないため、的外れです。
   \item[回答4]的外れではないです。このコメントは、動画で話された内容と関連していて、つるちゃん(鶴崎さん)に関する好意的な感想が述べられています。
   \item[回答5]的外れではないです。このコメントは、動画で紹介されたアイデアに対して称賛しているため、的外れではありません。
   \item[回答6]的外れです。このコメントは、動画で話された内容と全く関係ないため、的外れです。
   \item[回答7]的外れであると言えます。このコメントは、鶴崎さんの外見について言及しており、動画で話された内容とは関係がありません。
   \item[回答8]的外れではないです。このコメントは、伊沢さんのセンスを称賛しています。
   \item[回答9]的外れであると言えます。このコメントは、動画で話された内容とは関係がありません。
   \item[回答10]的外れであると言えます。このコメントは、動画で話された内容とは関係がありません。また、誤字というのは主観的な評価であるため、的外れと言えます。 
\end{description}
結果は全問正解であった.
\subsubsection{5〜7月}
レコメンデーション分野の論文調査を行なった.
しかし,ChatGPTによる分類比較は簡易的なものであったことと,大量のコメントをChatGPTに投げることがができないなどの問題から的外れを抽出する研究を継続することに決定した.またこの研究における的外れ抽出精度が悪かった場合,レビューから的外れを抽出するという引っ張るタスクの逆のタスクとしてレコメンデーションにシフトするという研究ストーリーも視野に入れていた.\par
的外れの体系化のため論文調査やデータセット収集を行い,教師なし学習による的外れ体系化を試みた.以下にk-means法によりクラスタ分けした結果を表\ref{para1}に示す.\\
\begin{table}[H]
  \caption{コメントデータセットを教師なし学習でクラスタリングを行なった結果の一部}
  \label{para1}
  \scriptsize
  \centering
  \begin{tabular}{|l|l|}
    \hline
    クラスタ番号& コメント \\
    \hline
    クラスタ2   & コメント : \\
    &ちゃんとナンバープレート凝ってるところすき笑笑      \\
    \hline
    クラスタ3 & コメント: \\
    &数字愛が強すぎる鶴崎さんかわいい    \\
     & コメント : \\
     & 最後にタクシー数で伊沢ナンバーのフラグを\\
     &きれいに回収するの、ショートなのに\\
     &満足感ありすぎ、秀逸すぎ、流石です\\
     &コメント : \\
     &伊沢さん情報てんこ盛りプレートなのに、\\
     &あまりにも違和感無さすぎる。\\
     &このプレートグッズにして欲しい\\     
    \hline
    クラスタ5&コメント :  \\
    &高校時代に数学偏差値98だったこうちゃんが\\
    &当て馬にされてるの草\\
    &コメント : \\
    &ふくらさん人間ぶってて草\\
    &コメント : \\
    &伊沢 タクシーw\\
    &516が誕生日とかちゃんと作ってあって好きw\\
    &コメント : \\
    &ちゃんと伊沢タクシーにしてあるのだいすき\\
  
    \hline
    \end{tabular}
\end{table}
今回はk-means法を用いてコメントのクラスタリングを行なったが,クラスタリングされたコメント間に明確な共通項を見つけることができなかった.しかしクラスタ5には動画固有の単語が含まれているコメントが多く集められている様子が見られた.これはword2vecの学習済みモデルを用いてベクトル化したため事前学習で得ることのできなかった固有の単語が集められていることが考えられる.クラスタ5に的外れでないものが集まっていることから,このようにその商品や動画特有の単語を集めることで的外れ抽出に活用できる可能性もあることがわかった.
\subsubsection{活動を通しての反省点と対策}
4月に行なったChatGPTによる分類の精度が高いことを受け,2ヶ月間別テーマについての調査を行なってきたため現在のテーマに関して進捗が少なかったことは反省点である.しかし別テーマの調査とはいえ関連テーマではあったためこの調査活動を通して得たレコメンデーション分野の知見は,現在のテーマの的外れ抽出に対しても有益であったと考えている.このようなテーマの変更は研究進捗の遅れを招くため極力避け,期待しない結果が得られてもその原因分析と追求を詳細に行うことが対策になると考えている.
\section{今後の活動計画}
8月からはデータセット収集と論文調査を並行して行いながら的外れの体系化を目指す.的外れの体系化が終わり次第ラベル検討を行いアノテーションを付与する.10月中にはこのデータセットを用いた基礎実験を行い,実験手法周りの基礎技術を習得する.\par
11月からは本実験に入り,12月は言語処理学会の原稿執筆に取り掛かり終わり次第,3月まで実験の性能向上を目指す.\par
月毎のスケジュールを以下に示す.
\begin{description}
   
   \item[8月〜10月]\mbox{}\\
	    データセット収集,論文調査,的外れ体系化,ラベル検討,アノテーション付与,基礎実験,実験手法周りの基礎技術習得.
   \item[11月]\mbox{}\\
	    実験
    \item[12月]\mbox{}\\
           実験,言語処理学会原稿執筆
    \item[1〜2月]\mbox{}\\
           実験
    \item[3月]\mbox{}\\
           言語処理学会発表
\end{description}


\begin{thebibliography}{99}
\bibitem{source1}消費者庁,平成29年版消費者白書第1部 第3章 第1節(4)若者の情報の活用や向き合い方 \url{https://www.caa.go.jp/policies/policy/consumer_research/white_paper/2017/white_paper_133.html#zuhyo-1-3-1-25}
\bibitem{paper1} 紀本雅大, 伊藤淳子, and 宗森純. ``評価表現に着目した映画レビューからの評価情報抽出.'' 研究報告デジタルコンテンツクリエーション (DCC) 2019.43 (2019): pp.1-8.
\bibitem{paper2} 小池惇爾, 松吉俊, and 福本文代. ``評価視点別レビュー要約のための重要文候補抽出.'' 言語処理学会第 18 回年次大会論文集 (2012): pp.1188-1191.
\bibitem{paper3} 小林 賢介,雨宮 佑基,酒井 哲也,``有用なレビューを抽出するための比較文フィルタリング",DEIM Forum 2021 H11-4.
\bibitem{paper4} 山澤美由紀,吉村宏樹,``Amazonレビュー文の有用性判別実験",情報処理学会研究報告自然言語処理(NL),53(2006-NL-173)号,pp.15-20.
\bibitem{paper5} 林 驍,伊東 栄典, ``大規模ECサイトの商品レビュー傾向分析'',	
研究報告知能システム(ICS),7(2015-ICS-181)号,pp.1-6.
\bibitem{paper6} 曽田 颯人,``商品レビューの複数の観点からの有用性の評価",JAIST Repository,\url{http://hdl.handle.net/10119/17106}
\bibitem{source2} QuizKnock,「理系レベル1と理系レベル99の違い」,YouTube,\url{https://www.youtube.com/shorts/OJ--u2fKTFU} (参照日 2023年1月9日).

\end{thebibliography}
\end{document}